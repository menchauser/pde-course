\chapter{Математическая модель}

Для данной задачи очевидным является записать математическую модель в
виде уравнения распространения электромагнитных волн.
\[
\left\{
\begin{array}{rclrcl}
  \frac{1}{c^2} \frac{\partial^2 E_x}{\partial t^2} &=& \Delta_{yz}
  E_x, & \Delta_{yz} &=& \frac{\partial^2}{\partial y^2} + \frac{\partial^2}{\partial z^2}\\
  E_x|_{z=0} &=& \sin\frac{\pi y}{l_y} \sin\frac{2\pi c}{\lambda}t, & E_x|_{z=l_z} &=& 0,\\
  E_x|_{y=0} &=& 0, & E_x|_{y=l_y} &=& 0,\\
  E_x|_{t=0} &=& 0, & \frac{\partial E_x}{\partial t} &=& 0.
\end{array}
\right.
\]
\\
Как видно, такую задачу невозможно решить методом разделения
переменных вследствие неоднородного краевого условия. Для решения этой
проблемы сделаем замену $U = E_x - \frac{l_z - z}{l_z} \sin\frac{2\pi
  c}{\lambda}t \sin\frac{\pi y}{l_y}$. Получим следующую задачу
математической физики
\[
\left\{
  \begin{array}{l}
    \frac{1}{c^2} \frac{\partial^2 U}{\partial t^2} = \Delta_{yz} U + \pi^2
    \frac{l_z - z}{lz}\frac{4l_y^2 - \lambda^2}{\lambda^2
      \l_y^2}\sin\frac{2\pi c}{\lambda}t \sin\frac{\pi y}{l_y},\\
    \begin{array}{rcl}
      U|_{z=0} &=& 0,\\
      U|_{y=0} &=& 0,\\
      U|_{t=0} &=& 0.\\
    \end{array}
    \begin{array}{rcl}
      U|_{z=l_z} &=& 0,\\
      U|_{y=l_y} &=& 0,\\
      \frac{\partial U}{\partial t} &=& \frac{2\pi
        c}{\lambda} \frac{l_z - z}{l_z} \sin\frac{\pi y}{\l_y}.
    \end{array}
  \end{array}
\right.
\]
\\
Получена неоднородная задача, имеющая однородные краевые условия (что
позволяет нам применять метод разделения переменных).
