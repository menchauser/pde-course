\chapter{Решение}

\section{Собственные функции}

Для последующих действий необходимо представить функцию $U$ в виде разложения по собственным функциям. Сначала найдем собственные функции. Для этого решим задачу Штурма-Лиувилля.\\
Представим функцию в виде
\[
U(y, z, t) = T(t)Y(y)Z(z).
\]
Подставим данное представление в следующее однородное уравнение
\[
\frac{1}{c^2} \frac{\partial^2 U}{\partial t^2} = \Delta_{yz} U\\
\]
И получим следующий вид
\[
\frac{1}{c^2}\frac{T''}{T} = \frac{Y''}{Y} + \frac{Z''}{Z} = -(\lambda^2 + \mu^2).\\
\]

Решив две задачи Штурма-Лиувилля
\[
\begin{array}{ll}
  \left\{
    \begin{array}{l}
      Y'' + \lambda^2Y = 0, \\
      Y|_{y=0} = Y|_{y=l_y} = 0,
    \end{array}
  \right.
  \left\{
    \begin{array}{l}
      Z'' + \mu^2Z = 0, \\
      Z|_{z=0} = Z|_{z=l_z} = 0,
    \end{array}
  \right.  
\end{array}
\]
получим, следующие собственные числа и собственные функции для $Y$ и $Z$
\[
\begin{array}{ll}
  \left\{
    \begin{array}{l}
      Y = A\sin\lambda y, \\
      \lambda = \frac{\pi n}{l_y},
    \end{array}
  \right.
  \left\{
    \begin{array}{l}
      Z = B\sin\mu z, \\
      \mu = \frac{\pi m}{l_z}.
    \end{array}
  \right.  
\end{array}
\]

\section{Поиск решения}

Таким образом, $U$ можно представить в виде следующего двойного ряда Фурье
\[
U(y, z, t) = \displaystyle \sum_{m=1}^{\infty}\sum_{n=1}^{\infty} \gamma(t) \sin\frac{\pi n y}{l_y} \sin\frac{\pi m z}{l_z}.
\]
\\
Разложим по данным собственным функциям неоднородную правую часть и все начальные условия. В дальнейших рассуждениях для простоты умножим левую и правую части уравнения на $c^2$, получив
\[
\frac{\partial^2 U}{\partial t^2} = \Delta_{yz} U + \pi^2 c^2
\frac{l_z - z}{lz}\frac{4l_y^2 - \lambda^2}{\lambda^2
  \l_y^2}\sin\frac{2\pi c}{\lambda}t \sin\frac{\pi y}{l_y}.\\
\]
\\
%Кроме того введем для простоты две замены
% \begin{eqnarray*}
%   k &=& \frac{2\pi c}{\lambda},\\
%   w_{nm} &=& \pi c \sqrt{\frac{n^2}{l_y^2} + \frac{m^2}{l_z^2}}.\\
% \end{eqnarray*}

\begin{enumerate}
\item Разложение неоднородной правой части.
  Разложим следующую функцию $G(y, z, t) = \pi^2 c^2\frac{l_z - z}{lz}\frac{4l_y^2 - \lambda^2}{\lambda^2 \l_y^2}\sin\frac{2\pi c}{\lambda}t \sin\frac{\pi y}{l_y}$ по данным базисным функциям. Заметим, что функция $G$ уже содержит в себе собственную функцию $\sin\frac{\pi y}{l_y}$, значит, необходимо разложить лишь зависящую от $z$ часть. Разложение будет иметь вид
  \[
  G(y, z, t) =  \pi^2 c^2 \frac{4l_y^2 - \lambda^2}{\lambda^2 \l_y^2} \sin\frac{2\pi c}{\lambda}t\displaystyle \sum_{m=1}^{\infty}  g_{nm}^{(z)}(t) \sin\frac{\pi y}{l_y} \sin\frac{\pi m z}{l_z}.
  \]
  Найдем коэффициенты ряда. 
  \begin{eqnarray*}
    g_{nm}(t) &=& \frac{2}{l_z} \displaystyle \int_0^{l_z} \frac{l_z - z}{l_z} \sin\frac{\pi m z}{l_z} dz = \frac{2}{l_z}\left[\int_0^{l_z}\sin\frac{\pi m z}{l_z}dz - \int_0^{l_z}\frac{z}{l_z} \sin\frac{\pi m z}{l_z}dz\right]\\
    &=& \frac{2}{l_z} \left[ \left. -\frac{l_z}{\pi m} \cos\frac{\pi m z}{l_z}\right|_{0}^{l_z} - \left. \frac{z l_z}{\pi m}\cos\frac{\pi m z}{l_z}\right|_{0}^{l_z} + \int_0^{l_z}\frac{l_z}{\pi m}\cos\frac{\pi m z}{l_z}dz\right]\\
    &=& \frac{2}{l_z}\frac{l_z}{\pi m} \left( 1 - (-1)^m - (-1)^{m+1} \right) = \frac{2}{\pi m}.
  \end{eqnarray*}
  Таким образом, получаем
  \[
  G(y, z, t) = 2\pi c^2\frac{4l_y^2 - \lambda^2}{\lambda^2 \l_y^2} \sin\frac{2\pi c}{\lambda}t\displaystyle \sum_{m=1}^{\infty}\frac{1}{m}\sin\frac{\pi y}{l_y} \sin\frac{\pi m z}{l_z}.
  \]

\item Разложение начального условия.


  Для удобства введем замену $k = \frac{2\pi c}{\lambda}$. Теперь разложим начальное условие  $\Phi(y, z, t) = -k\frac{l_z - z}{l_z}\sin\frac{\pi y}{l_y}$. Как и в предыдущем случае, она уже разложена по собственным функциям относительно $y$, будем искать разложение относительно $sin\frac{\pi m z}{l_z}$. Разложения будет иметь следующий вид
  \[
  \Phi(y, z, t) = \displaystyle -k\sum_{m=1}^{\infty}\varphi(t) \sin\frac{\pi y}{l_y} \sin\frac{\pi m z}{l_z}.
  \]
  Найдем коэффициенты разложения
  \begin{eqnarray*}
    \varphi(t) = \displaystyle \frac{2}{l_z} \int_0^{l_z} \frac{l_z - z}{l_z}\sin\frac{\pi m z}{l_z}dz = \frac{2}{\pi m}.
  \end{eqnarray*}
  Таким образом
  \[
  \Phi(y, z, t) = -\frac{2k}{\pi}\displaystyle \sum_{m=1}^{\infty} \frac{1}{m} \sin\frac{\pi y}{l_y} \sin\frac{\pi m z}{l_z}.
  \]
\end{enumerate}

Подставим все полученные разложения и учтем, что правая часть и начальное условие не являются нулевыми лишь при $n = 1$, поэтому и разложение $U(y, z, t)$ представим в виде обыкновенного ряда Фурье.
\[
U(y, z, t) = \displaystyle \sum_{m=1}^{\infty} \gamma(t) \sin\frac{\pi y}{l_y} \sin\frac{\pi m z}{l_z}.
\]
Подставив в систему разложения и приравняв коэффициенты при одинаковых базисных функциях, получим систему обыкновенных дифференциальных уравнений с начальными условиями вида
\[
\left\{
    \begin{array}{l}
      \gamma''(t) + w^2_{m}\gamma(t) = \frac{2\pi c^2}{m}\left(\frac{4l_y^2 - \lambda^2}{l_y^2\lambda^2} \right)\sin{kt},\\
      \begin{array}{rcl}
      \gamma(0) &=& 0,\\
      \gamma'(0) &=& -\frac{2k}{\pi m}.
      \end{array}
    \end{array}
\right.
\]\\
Здесь $w_{m} = \pi c \sqrt{\frac{1}{l_y^2} + \frac{m^2}{l_z^2}}$\\
Найдем общее решение соответствующего однородного уравнения
\[
\begin{array}{l}
  \gamma''(t) + w^2_{m}\gamma(t) = 0,\\
  \gamma^{0}(t) = A\cos{w_m t} + B\sin{w_mt}.
\end{array}
\]
Рассмотрим теперь неоднородную задачу и найдем ее частное решение. Исходя из вида правой части, вид решения будет таким
\[
\tilde{\gamma}(t) = C\cos{kt} + D\sin{kt}.
\]
Подставим функцию такого вида в уравнение и получим
\[
-k^2C\cos{kt} - k^2 D\sin{kt} + w^2_m C\cos{kt} + w^2_{m} D\cos{kt} = \frac{2\pi c^2}{m}\left(\frac{4l_y^2 - \lambda^2}{l_y^2\lambda^2} \right)\sin{kt}.
\]
Приравняем коэффициенты при соответствующих функциях и получим систему
$\begin{equationsset}
      (w_m^2 - k^2)C & = & 0, \\
      (w_m^2 - k^2)D & = & \frac{2\pi c^2}{m}\left(\frac{4l_y^2 - \lambda^2}{l_y^2\lambda^2} \right).
\end{equationsset}$
      

Отсюда $C = 0$, $D = \frac{2\pi c^2}{m(w_m^2 - k^2)}\left(\frac{4l_y^2 - \lambda^2}{l_y^2\lambda^2} \right)$.\\
Общее решение выглядит как сумма общего решения однородного уравнения и частного решения неоднородного, то есть
\[
\gamma(t) = A\cos{w_mt} + B\sin{w_mt} + \frac{2\pi c^2}{m(w_m^2 - k^2)}\left(\frac{4l_y^2 - \lambda^2}{l_y^2\lambda^2} \right)\sin{kt}.
\]
Используем начальные условия и получим систему
\[
\left\{
  \begin{array}{l}
    A = 0,\\
    w_nB + kD = -\frac{2k}{\pi m} \Rightarrow B = -\frac{k}{w_m}\left(\frac{2}{\pi m} + D \right).
  \end{array}
\right.
\]
Таким образом,
\begin{eqnarray*}
  \gamma(t) &=& \frac{2\pi c^2}{m(w_m^2 - k^2)}\left(\frac{4l_y^2 - \lambda^2}{l_y^2\lambda^2} \right)\sin{kt} - \frac{k}{w_m}\left(\frac{2}{\pi m} + D\right)\sin{w_mt},\\
  U(y, z, t) &=& \displaystyle \sum_{m=1}^{\infty} \left( \frac{2\pi c^2}{m(w_m^2 - k^2)}\left(\frac{4l_y^2 - \lambda^2}{l_y^2\lambda^2} \right)\sin{kt} - \frac{k}{w_m}\left(\frac{2}{\pi m} + D\right)\sin{w_mt} \right) \sin\frac{\pi y}{l_y} \sin\frac{\pi m z}{l_z},\\
  E_x(y, z, t) &=& \displaystyle
    \sum_{m=1}^{\infty} \left( \frac{2\pi c^2}{m(w_m^2 - k^2)}\left(\frac{4l_y^2 - \lambda^2}{l_y^2\lambda^2} \right)\sin{kt} - \frac{k}{w_m}\left(\frac{2}{\pi m} + D\right)\sin{w_mt} \right) \sin\frac{\pi y}{l_y} \sin\frac{\pi m z}{l_z}\\
    &+& \frac{l_z - z}{l_z} \sin\frac{2\pi c}{\lambda}t \sin\frac{\pi y}{l_y}
\end{eqnarray*}
