\chapter*{\begin{center}Реферат\end{center}}

\textbf{Пояснительная записка:} \ref{TotPages} с., 1 таблица, 3 источника.
\vspace{1cm}

УРАВНЕНИЯ МАТЕМАТИЧЕСКОЙ ФИЗИКИ, КРАЕВАЯ ЗАДАЧА, \\УРАВНЕНИЕ ЭЛЕКТРОМАГНИТНЫХ ВОЛН,
РЯД ФУРЬЕ ПО ТРИГОНОМЕТРИЧЕСКИМ ФУНКЦИЯМ, ИССЛЕДОВАНИЕ СХОДИМОСТИ РЯДА, \\ОЦЕНКА ОСТАТКА РЯДА
\vspace{1cm}

Объектом исследования является метод Фурье решения краевых задач.

Цель работы~---  нахождение решения краевой задачи в виде ряда Фурье по собственным функциям,
исследование его сходимости и оценка остатка.

Разработана компьютерная программа, реализующая суммирование элементов полученного ряда с 
заданной точностью. С помощью данной программы проведено исследование качества оценки остатка
ряда.

