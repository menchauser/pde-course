\chapter*{Введение}
\addcontentsline{toc}{chapter}{\tocsecindent{Введение}}

В данной работе решается краевая задача математической физики, основанная на волновом уравнении, и программно моделируется распространение поляризованной электромагнитной волны в плоскопараллельном однородном волноводе. Подобные задачи возникают при исследовании электромагнитных волн, построении волноводов, передающих сигналы.

Для их решения применяются различные методы: аналитические, численные, вариационные, проекционные.  В данной работе рассматриваются аналитические методы, в частности для решения предложенной задачи используется метод Фурье разделения переменных. Этот метод выбран ввиду того, что сама задача линейна относительно частных производных с постоянными коэффициентами, а метод Фурье прост для понимания и предназначен как раз для решения таких задач.

Схема решения заключается в следующем: искомая функция факторизуется по каждому своему параметру, затем решается задача Штурма-Лиувилля и получаются собственные функции оператора Лапласа. Далее решение ищется в виде ряда по этим собственным функциям.

Для наглядности написана программа, анимирующая распространение электромагнитной волны в волноводе на основе полученного решения. Кроме того, получена равномерная оценка погрешности вычислений и проведено исследование её эффективности.